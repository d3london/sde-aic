% Options for packages loaded elsewhere
\PassOptionsToPackage{unicode}{hyperref}
\PassOptionsToPackage{hyphens}{url}
\PassOptionsToPackage{dvipsnames,svgnames,x11names}{xcolor}
%
\documentclass[
  letterpaper,
  DIV=11,
  numbers=noendperiod]{scrartcl}

\usepackage{amsmath,amssymb}
\usepackage{iftex}
\ifPDFTeX
  \usepackage[T1]{fontenc}
  \usepackage[utf8]{inputenc}
  \usepackage{textcomp} % provide euro and other symbols
\else % if luatex or xetex
  \usepackage{unicode-math}
  \defaultfontfeatures{Scale=MatchLowercase}
  \defaultfontfeatures[\rmfamily]{Ligatures=TeX,Scale=1}
\fi
\usepackage{lmodern}
\ifPDFTeX\else  
    % xetex/luatex font selection
\fi
% Use upquote if available, for straight quotes in verbatim environments
\IfFileExists{upquote.sty}{\usepackage{upquote}}{}
\IfFileExists{microtype.sty}{% use microtype if available
  \usepackage[]{microtype}
  \UseMicrotypeSet[protrusion]{basicmath} % disable protrusion for tt fonts
}{}
\makeatletter
\@ifundefined{KOMAClassName}{% if non-KOMA class
  \IfFileExists{parskip.sty}{%
    \usepackage{parskip}
  }{% else
    \setlength{\parindent}{0pt}
    \setlength{\parskip}{6pt plus 2pt minus 1pt}}
}{% if KOMA class
  \KOMAoptions{parskip=half}}
\makeatother
\usepackage{xcolor}
\setlength{\emergencystretch}{3em} % prevent overfull lines
\setcounter{secnumdepth}{-\maxdimen} % remove section numbering
% Make \paragraph and \subparagraph free-standing
\ifx\paragraph\undefined\else
  \let\oldparagraph\paragraph
  \renewcommand{\paragraph}[1]{\oldparagraph{#1}\mbox{}}
\fi
\ifx\subparagraph\undefined\else
  \let\oldsubparagraph\subparagraph
  \renewcommand{\subparagraph}[1]{\oldsubparagraph{#1}\mbox{}}
\fi


\providecommand{\tightlist}{%
  \setlength{\itemsep}{0pt}\setlength{\parskip}{0pt}}\usepackage{longtable,booktabs,array}
\usepackage{calc} % for calculating minipage widths
% Correct order of tables after \paragraph or \subparagraph
\usepackage{etoolbox}
\makeatletter
\patchcmd\longtable{\par}{\if@noskipsec\mbox{}\fi\par}{}{}
\makeatother
% Allow footnotes in longtable head/foot
\IfFileExists{footnotehyper.sty}{\usepackage{footnotehyper}}{\usepackage{footnote}}
\makesavenoteenv{longtable}
\usepackage{graphicx}
\makeatletter
\def\maxwidth{\ifdim\Gin@nat@width>\linewidth\linewidth\else\Gin@nat@width\fi}
\def\maxheight{\ifdim\Gin@nat@height>\textheight\textheight\else\Gin@nat@height\fi}
\makeatother
% Scale images if necessary, so that they will not overflow the page
% margins by default, and it is still possible to overwrite the defaults
% using explicit options in \includegraphics[width, height, ...]{}
\setkeys{Gin}{width=\maxwidth,height=\maxheight,keepaspectratio}
% Set default figure placement to htbp
\makeatletter
\def\fps@figure{htbp}
\makeatother

\KOMAoption{captions}{tableheading}
\makeatletter
\@ifpackageloaded{caption}{}{\usepackage{caption}}
\AtBeginDocument{%
\ifdefined\contentsname
  \renewcommand*\contentsname{Table of contents}
\else
  \newcommand\contentsname{Table of contents}
\fi
\ifdefined\listfigurename
  \renewcommand*\listfigurename{List of Figures}
\else
  \newcommand\listfigurename{List of Figures}
\fi
\ifdefined\listtablename
  \renewcommand*\listtablename{List of Tables}
\else
  \newcommand\listtablename{List of Tables}
\fi
\ifdefined\figurename
  \renewcommand*\figurename{Figure}
\else
  \newcommand\figurename{Figure}
\fi
\ifdefined\tablename
  \renewcommand*\tablename{Table}
\else
  \newcommand\tablename{Table}
\fi
}
\@ifpackageloaded{float}{}{\usepackage{float}}
\floatstyle{ruled}
\@ifundefined{c@chapter}{\newfloat{codelisting}{h}{lop}}{\newfloat{codelisting}{h}{lop}[chapter]}
\floatname{codelisting}{Listing}
\newcommand*\listoflistings{\listof{codelisting}{List of Listings}}
\makeatother
\makeatletter
\makeatother
\makeatletter
\@ifpackageloaded{caption}{}{\usepackage{caption}}
\@ifpackageloaded{subcaption}{}{\usepackage{subcaption}}
\makeatother
\ifLuaTeX
  \usepackage{selnolig}  % disable illegal ligatures
\fi
\usepackage{bookmark}

\IfFileExists{xurl.sty}{\usepackage{xurl}}{} % add URL line breaks if available
\urlstyle{same} % disable monospaced font for URLs
\hypersetup{
  pdftitle={London SDE/AIC Programme: Introduction and Proposed Use-Cases},
  colorlinks=true,
  linkcolor={blue},
  filecolor={Maroon},
  citecolor={Blue},
  urlcolor={Blue},
  pdfcreator={LaTeX via pandoc}}

\title{London SDE/AIC Programme: Introduction and Proposed Use-Cases}
\author{Dr.~Joe Zhang \emph{(Head of Data Science)} \and Prof.~James Teo
\emph{(Clinical Director AI \& Data)} \and Dr.~Jorge Cardoso
\emph{(Chief Technology Officer)} \and Jawad Chaudhry \emph{(Programme
Lead)} \and Sigal Hachlili \emph{(Director of AI, Data \& Digital)}}
\date{}

\begin{document}
\maketitle

\emph{Version 0.6 (last updated 2024 Apr 18)}

\subsection{Introduction}\label{introduction}

The \href{https://www.aicentre.co.uk/}{London AI Centre} (AIC) has been
commissioned as part of the London Secure Data Environment (SDE)
programme for its latest phase: to extend AI technologies and analytics
capabilities to stakeholders and data environments across London. This
document summarises the latest state of planning for the programme, as
an aid to internal and external stakeholders including Integrated Care
Boards (ICB) and the wider London NHS ecosystem.

\subsection{What is the London SDE?}\label{what-is-the-london-sde}

The London Secure Data Environment (SDE) is part of a national programme
to enable secure and more powerful analytics for NHS, academic, and
commercial users. Uniquely amongst regional peers, the London SDE does
not focus on a single research platform. Rather, it places a focus on
developing data infrastructure and capabilities across the region to
support population health, care providers, and commissioners. This is in
addition to building data environments that enable commercial research
and development partnerships.

The SDE is led by \textbf{OneLondon}, as part of an overarching London
Health Data Strategy, coalescing around three components
(Figure~\ref{fig-sde-summary}):

\begin{enumerate}
\def\labelenumi{(\arabic{enumi})}
\item
  \textbf{London Data Service (LDS)}: hosted in North-East London, the
  LDS serves as a data engineering and service layer for pan-London
  primary care and secondary care data. It handles data extraction and
  linkage, and provisions data within secure analytics environments for
  both research and NHS users.
\item
  \textbf{DiscoverNOW Research/Analytics Environment}: run by Imperial
  College Healthcare Partners in North-West London, DiscoverNOW supports
  governance and operation of secure research environments for academic,
  commercial, and NHS research and analytics.
\item
  \textbf{London AI Centre (AIC)}: a national centre of excellence for
  applied data science and AI, the AIC provides frontier technology for
  data enrichment (CogStack), federated analytics (FLIP), and deployment
  of machine learning tools, as well as expertise in health data and
  advanced analytics.
\end{enumerate}

\begin{figure}

\centering{

\includegraphics[width=6in,height=4.14in]{index_files/figure-latex/mermaid-figure-6.png}

}

\caption{\label{fig-sde-summary}Summary of SDE components and data
flows. Each London ICB is provisioned with its own data/analytics
environment through the LDS. FLIP = Federated Learning and
Interoperability Platform.}

\end{figure}%

\textsubscript{Source:
\href{https://d3london.github.io/sde_aic_docs/index.qmd.html}{Article
Notebook}}

\subsection{Technology and objectives}\label{technology-and-objectives}

The contribution from the London AIC consists of technology deployment
and supporting expertise, that enable a number of objectives
(Figure~\ref{fig-aic-objectives}) over the two year programme. This
contribution includes the following:

\begin{enumerate}
\def\labelenumi{(\arabic{enumi})}
\item
  \textbf{Federated Learning and Interoperability Platform (FLIP)}: FLIP
  consists of (a) secure data environments within NHS hospital Trusts
  for multi-modal imaging data, imaging metadata, and structured health
  record data in the OMOP common data model; and (b) a mechanism to
  query data and train AI models across these secure enclaves without
  the need to physically transfer data. FLIP is presently installed in
  four major London Trusts. Integrating FLIP into the SDE will enable
  hospital data (such as cancer data) to be surfaced into the LDS, and
  enable access to multi-modal data (such as DICOM imaging and digital
  pathology) for research in precision healthcare.
\item
  \textbf{CogStack}: As an advanced natural language processing
  platform, CogStack can turn the large quantities of health information
  that are found in narrative text, into structured and analysable data.
  Currently actively used in Trusts to assist with clinical coding from
  notes and clinic letters, CogStack can surface secondary care and
  cancer pathway data, and previously unseen primary care data, into the
  SDE ecosystem.
\item
  \textbf{AIC Data/AI Hub}: The AIC hosts substantial health data and AI
  implementation expertise, that will provide practical support in data
  engineering, clinical informatics, data science, and machine learning
  (ML) development and deployment. Primary aims are to (a) help
  Integrated Care Boards (ICB) migrate data pipelines and analytics into
  common data models and terminologies within LDS environments; (b)
  extend these into reproducible pipelines for data science and
  predictive analytics deployment; and (c) work together to make ICBs
  self-sufficient in these capabilities. The AIC will also support the
  adoption and roll-out of the OMOP Common Data Model.
\end{enumerate}

\begin{figure}

\centering{

\includegraphics[width=6in,height=3.38in]{index_files/figure-latex/mermaid-figure-10.png}

}

\caption{\label{fig-aic-objectives}Summary of AIC work components and
objectives. FLIP = Federated Learning and Interoperability Platform; ML
= Machine Learning.}

\end{figure}%

\textsubscript{Source:
\href{https://d3london.github.io/sde_aic_docs/index.qmd.html}{Article
Notebook}}

As the LDS ICB environments share a common data model, any pipelines
created in collaboration with an ICB can be adapted and used for any
other ICB (or deployed across multiple environments to create pan-London
insights). This will also facilitate the use of shared terminologies,
and validating / versioning / serving NHS-owned machine learning models
across regions.

\subsection{Proposed use-cases}\label{proposed-use-cases}

The following use-cases are \emph{examples} of analytics projects that
can be supported within the SDE ecosystem, in collaboration between
ICB/NHS analytics teams and the AIC/SDE team. Use-cases align to the
London Health Data Strategy and long term condition priorities, as well
as national programmes such as CORE20PLUS5, and are proposed here
following early discussions with London ICBs. An overarching objective
for any work is to build a foundation for reproducible analytics that
can be shared across the region.

\subsubsection{Systematic measurement of group and individual health
inequality}\label{systematic-measurement-of-group-and-individual-health-inequality}

\textbf{AIM:} To systematically surface multiple dimensions of health
inequality across sociodemographic / geospatial groups and individual
patients, and to monitor this data continuously across key long-term
conditions.

\textbf{BACKGROUND:} Health inequality refers to measurable differences
in health outcomes and determinants between individuals or groups
(e.g.~morbidity, co-morbidity, disease complications/death, healthcare
access, disease screening, treatment delivery). Where individuals and
groups experience health inequality, the principle of health
\emph{equity} emphases the importance of reducing disparities by
modifying outcome determinants that are unfairly distributed.

Health inequality is traditionally measured and visualised as a
comparison of prevalence/incidence across different population groups.
While helpful for broad insights, this offers limited understanding of
complex individual circumstances. This type of measurement can be
extended to individual patients, by using clinical domain knowledge to
define `indicators' of unequal disease, diagnosis, and treatment
pathways. For example, in an individual with Diabetes Mellitus,
indicators of inequality can include:

\begin{enumerate}
\def\labelenumi{(\arabic{enumi})}
\item
  Diabetes surfacing at an early age;
\item
  Diagnosis in proximity to cardiovascular risk factor co-morbidities;
\item
  Diagnosis at a \emph{late} age but with more severe disease, as
  measured by HbA1c or presence of end-organ complications;
\item
  Reduced health engagement/encounters/treatment compared to what is
  expected based on disease severity;
\item
  Shorter time to complications and mortality following diagnosis.
\end{enumerate}

The precise contribution of factors to outcomes can be measured and
understood in multivariate statistical models. Overall, the presence and
magnitude of indicators can be used to visualise, monitor, and explain
different types of inequality, including through comparison of groups
and individuals to `what is expected' in a background population. The
outcome is an increase in actionability, with identification of
modifiable determinants of inequality ( = inequity) for small groups and
individuals.

\textbf{APPROACH:} Broadly, work would fall into three stages. The first
includes defining shared terminologies, concepts, and indicators that
cover long-term conditions of interest. Secondly, existing descriptions
of health inequality can be migrated onto the LDS environment using
shared terminologies and concepts, such that any condition can be
repreducibly visualised across multiple dimensions and `cuts'.

Finally, this work would be extended to encompass specific inequality
indicators and statistical insights, at a small group and individual
level. (Figure~\ref{fig-rep-pipelines-inequality}) shows outcomes in an
example workflow for long-term conditions (not including cancer).

\begin{figure}

\centering{

\includegraphics[width=6in,height=1.29in]{index_files/figure-latex/mermaid-figure-9.png}

}

\caption{\label{fig-rep-pipelines-inequality}Outcomes in development
workflow. LTC = long-term conditions.}

\end{figure}%

\textsubscript{Source:
\href{https://d3london.github.io/sde_aic_docs/index.qmd.html}{Article
Notebook}}

The primary output of this project would be a code base that engineers
cohorts from disease definitions, produces indicators for a given
disease, and produces summary tables and statistics for groups and
individual patients (where required). The code can be adapted by ICBs
and used to support local dashboards and pathways. Code can be used for
higher-level interval reporting and monitoring for the London region.

\subsubsection{Cardiovascular disease prevention through decision
intelligence}\label{cardiovascular-disease-prevention-through-decision-intelligence}

\textbf{AIM:} To enhance descriptive population health management with
explainable predictive analytics and clinical guideline-based decision
intelligence systems, across cardiovascular related co-morbidities
(including hypertension, diabetes, chronic kidney disease).

\textbf{BACKGROUND:} The spectrum of cardiovascular long-term conditions
and associated risk factors is wide, and includes hypertension,
diabetes, obesity, high cholesterol, ischaemic heart disease, stroke,
and chronic kidney disease, as well as dementia and heart failure. The
burden of such diseases is high.
\href{https://cks.nice.org.uk/topics/cvd-risk-assessment-management/background-information/burden-of-cvd/}{Heart
disease} alone causes a quarter of deaths in the UK, with direct costs
to the healthcare system estimated at £9 billion by the British Heart
Foundation. Cardiovascular disease is seen as a
\href{https://imperialcollegehealthpartners.com/portfolio/onelondon/}{priority
area for use of data} across OneLondon patient and public engagement.

There is robust aggregate understanding of cardiovascular long-term
conditions in London, through prevalence reporting and Quality Outcome
Framework (QOF) indicators. Existing ICB dashboards
(Figure~\ref{fig-icb-hypertension}) routinely show how a practice or a
system are performing relative to their peers. However, while indicating
priority areas for action, such reporting has limitations. These include
inability to surface individual patients and/or direct actions, lack of
adjustment for demographics and other variables, and consideration of
long-term conditions in isolation (whereas multi-morbidity changes the
entire risk profile and urgency of response for individuals).

\begin{figure}

\centering{

\includegraphics{media/example_dashboard.jpg}

}

\caption{\label{fig-icb-hypertension}Existing ICB dashboard for
Hypertension}

\end{figure}%

Some of these limitations are being addressed by existing work in London
pathfinder programmes, and in other regions such as Greater Manchester,
which are moving towards electronic identification of patients who may
be actioned via pre-agreed clinical pathways
(Figure~\ref{fig-simple-pathway-action}).

\begin{figure}

\centering{

\includegraphics[width=6in,height=2.51in]{index_files/figure-latex/mermaid-figure-8.png}

}

\caption{\label{fig-simple-pathway-action}Examples of simple logical
triggers leading to clinical actions. CKD = Chronic Kidney Disease; BB =
Beta-blocker; ACEi = ACE inhibitor.}

\end{figure}%

\textsubscript{Source:
\href{https://d3london.github.io/sde_aic_docs/index.qmd.html}{Article
Notebook}}

A previous collaboration between the AIC and North-East London ICB was
able to develop precise cardiovascular risk prediction models for
individuals, using explainable machine-learning algorithms and the
linked patient health record. Actionable factors could also be
highlighted in patients with high risk, with their relative importance
explained through statistical modelling to enhance explainability
(Figure~\ref{fig-htn-actionable}).

\begin{figure}

\centering{

\includegraphics{media/htn_actionable.jpg}

}

\caption{\label{fig-htn-actionable}Actionable factors (including
follow-up, treatment, blood pressure control) and association of
features with adverse outcome in high risk hypertensive patients}

\end{figure}%

\textbf{APPROACH:} This use-case will extend the above work, by
combining multivariate statistics and machine learning for risk
prediction, with robust decision systems that are grounded in evidence
and clinical guidelines. Broadly, work would consist of the following
components:

\begin{enumerate}
\def\labelenumi{(\arabic{enumi})}
\item
  Development of shared terminologies, concepts, and features, used to
  characterise patients with any single or combination of relevant
  long-term condition.
\item
  Development of shared code base to ingest these definitions, and
  construct / describe / visualise cohorts as extension of existing
  dashboards. This code can be built as part of migration of existing
  pipelines into the LDS environment.
\item
  Computerisation of Quality Outcomes Framework targets and clinical
  guidelines, in conjunction with local clinical teams, to develop safe
  decision logic for use in the ``effector'' arm.
\item
  Use of CogStack to extract additional valuable context and missing
  codes from unstructured text.
\item
  Development of statistical and machine learning models for predicting
  and understanding risk of progression across range of cardiovascular
  morbidity and co-morbidity.
\item
  For given patient's health record, understand actions (i.e.~are there
  actions available, and what are they) combined with explainable risks
  across multiple conditions (i.e.~what are the highest risks for this
  patient and why).
\end{enumerate}

This approach aims to generate \textbf{patient-centric decision
intelligence} (Figure~\ref{fig-patient-dec-int}), where risk and
possible actions are considered for a highly detailed representation of
an individual, rather than for isolated conditions, or for patients to
be considered in large, aggregate groups. Any systems will need to be
evaluated and monitored for safety and fairness, with a process of
training and handover to continuity teams following the end of this SDE
programme phase.

\begin{figure}

\centering{

\includegraphics[width=6in,height=3.83in]{index_files/figure-latex/mermaid-figure-7.png}

}

\caption{\label{fig-patient-dec-int}A linked and enriched patient
healthcare record can support accurate risk prediction and
prioritisation of useful actions.}

\end{figure}%

\textsubscript{Source:
\href{https://d3london.github.io/sde_aic_docs/index.qmd.html}{Article
Notebook}}

\subsubsection{Joining up cancer
pathways}\label{joining-up-cancer-pathways}

\textbf{AIM:}

\textbf{BACKGROUND:}

\textbf{APPROACH:}

\subsection{Next steps}\label{next-steps}

\ldots{}



\end{document}
